
% Default to the notebook output style

    


% Inherit from the specified cell style.




    
\documentclass[11pt]{article}

    
    
    \usepackage[T1]{fontenc}
    % Nicer default font (+ math font) than Computer Modern for most use cases
    \usepackage{mathpazo}

    % Basic figure setup, for now with no caption control since it's done
    % automatically by Pandoc (which extracts ![](path) syntax from Markdown).
    \usepackage{graphicx}
    % We will generate all images so they have a width \maxwidth. This means
    % that they will get their normal width if they fit onto the page, but
    % are scaled down if they would overflow the margins.
    \makeatletter
    \def\maxwidth{\ifdim\Gin@nat@width>\linewidth\linewidth
    \else\Gin@nat@width\fi}
    \makeatother
    \let\Oldincludegraphics\includegraphics
    % Set max figure width to be 80% of text width, for now hardcoded.
    \renewcommand{\includegraphics}[1]{\Oldincludegraphics[width=.8\maxwidth]{#1}}
    % Ensure that by default, figures have no caption (until we provide a
    % proper Figure object with a Caption API and a way to capture that
    % in the conversion process - todo).
    \usepackage{caption}
    \DeclareCaptionLabelFormat{nolabel}{}
    \captionsetup{labelformat=nolabel}

    \usepackage{adjustbox} % Used to constrain images to a maximum size 
    \usepackage{xcolor} % Allow colors to be defined
    \usepackage{enumerate} % Needed for markdown enumerations to work
    \usepackage{geometry} % Used to adjust the document margins
    \usepackage{amsmath} % Equations
    \usepackage{amssymb} % Equations
    \usepackage{textcomp} % defines textquotesingle
    % Hack from http://tex.stackexchange.com/a/47451/13684:
    \AtBeginDocument{%
        \def\PYZsq{\textquotesingle}% Upright quotes in Pygmentized code
    }
    \usepackage{upquote} % Upright quotes for verbatim code
    \usepackage{eurosym} % defines \euro
    \usepackage[mathletters]{ucs} % Extended unicode (utf-8) support
    \usepackage[utf8x]{inputenc} % Allow utf-8 characters in the tex document
    \usepackage{fancyvrb} % verbatim replacement that allows latex
    \usepackage{grffile} % extends the file name processing of package graphics 
                         % to support a larger range 
    % The hyperref package gives us a pdf with properly built
    % internal navigation ('pdf bookmarks' for the table of contents,
    % internal cross-reference links, web links for URLs, etc.)
    \usepackage{hyperref}
    \usepackage{longtable} % longtable support required by pandoc >1.10
    \usepackage{booktabs}  % table support for pandoc > 1.12.2
    \usepackage[inline]{enumitem} % IRkernel/repr support (it uses the enumerate* environment)
    \usepackage[normalem]{ulem} % ulem is needed to support strikethroughs (\sout)
                                % normalem makes italics be italics, not underlines
    

    
    
    % Colors for the hyperref package
    \definecolor{urlcolor}{rgb}{0,.145,.698}
    \definecolor{linkcolor}{rgb}{.71,0.21,0.01}
    \definecolor{citecolor}{rgb}{.12,.54,.11}

    % ANSI colors
    \definecolor{ansi-black}{HTML}{3E424D}
    \definecolor{ansi-black-intense}{HTML}{282C36}
    \definecolor{ansi-red}{HTML}{E75C58}
    \definecolor{ansi-red-intense}{HTML}{B22B31}
    \definecolor{ansi-green}{HTML}{00A250}
    \definecolor{ansi-green-intense}{HTML}{007427}
    \definecolor{ansi-yellow}{HTML}{DDB62B}
    \definecolor{ansi-yellow-intense}{HTML}{B27D12}
    \definecolor{ansi-blue}{HTML}{208FFB}
    \definecolor{ansi-blue-intense}{HTML}{0065CA}
    \definecolor{ansi-magenta}{HTML}{D160C4}
    \definecolor{ansi-magenta-intense}{HTML}{A03196}
    \definecolor{ansi-cyan}{HTML}{60C6C8}
    \definecolor{ansi-cyan-intense}{HTML}{258F8F}
    \definecolor{ansi-white}{HTML}{C5C1B4}
    \definecolor{ansi-white-intense}{HTML}{A1A6B2}

    % commands and environments needed by pandoc snippets
    % extracted from the output of `pandoc -s`
    \providecommand{\tightlist}{%
      \setlength{\itemsep}{0pt}\setlength{\parskip}{0pt}}
    \DefineVerbatimEnvironment{Highlighting}{Verbatim}{commandchars=\\\{\}}
    % Add ',fontsize=\small' for more characters per line
    \newenvironment{Shaded}{}{}
    \newcommand{\KeywordTok}[1]{\textcolor[rgb]{0.00,0.44,0.13}{\textbf{{#1}}}}
    \newcommand{\DataTypeTok}[1]{\textcolor[rgb]{0.56,0.13,0.00}{{#1}}}
    \newcommand{\DecValTok}[1]{\textcolor[rgb]{0.25,0.63,0.44}{{#1}}}
    \newcommand{\BaseNTok}[1]{\textcolor[rgb]{0.25,0.63,0.44}{{#1}}}
    \newcommand{\FloatTok}[1]{\textcolor[rgb]{0.25,0.63,0.44}{{#1}}}
    \newcommand{\CharTok}[1]{\textcolor[rgb]{0.25,0.44,0.63}{{#1}}}
    \newcommand{\StringTok}[1]{\textcolor[rgb]{0.25,0.44,0.63}{{#1}}}
    \newcommand{\CommentTok}[1]{\textcolor[rgb]{0.38,0.63,0.69}{\textit{{#1}}}}
    \newcommand{\OtherTok}[1]{\textcolor[rgb]{0.00,0.44,0.13}{{#1}}}
    \newcommand{\AlertTok}[1]{\textcolor[rgb]{1.00,0.00,0.00}{\textbf{{#1}}}}
    \newcommand{\FunctionTok}[1]{\textcolor[rgb]{0.02,0.16,0.49}{{#1}}}
    \newcommand{\RegionMarkerTok}[1]{{#1}}
    \newcommand{\ErrorTok}[1]{\textcolor[rgb]{1.00,0.00,0.00}{\textbf{{#1}}}}
    \newcommand{\NormalTok}[1]{{#1}}
    
    % Additional commands for more recent versions of Pandoc
    \newcommand{\ConstantTok}[1]{\textcolor[rgb]{0.53,0.00,0.00}{{#1}}}
    \newcommand{\SpecialCharTok}[1]{\textcolor[rgb]{0.25,0.44,0.63}{{#1}}}
    \newcommand{\VerbatimStringTok}[1]{\textcolor[rgb]{0.25,0.44,0.63}{{#1}}}
    \newcommand{\SpecialStringTok}[1]{\textcolor[rgb]{0.73,0.40,0.53}{{#1}}}
    \newcommand{\ImportTok}[1]{{#1}}
    \newcommand{\DocumentationTok}[1]{\textcolor[rgb]{0.73,0.13,0.13}{\textit{{#1}}}}
    \newcommand{\AnnotationTok}[1]{\textcolor[rgb]{0.38,0.63,0.69}{\textbf{\textit{{#1}}}}}
    \newcommand{\CommentVarTok}[1]{\textcolor[rgb]{0.38,0.63,0.69}{\textbf{\textit{{#1}}}}}
    \newcommand{\VariableTok}[1]{\textcolor[rgb]{0.10,0.09,0.49}{{#1}}}
    \newcommand{\ControlFlowTok}[1]{\textcolor[rgb]{0.00,0.44,0.13}{\textbf{{#1}}}}
    \newcommand{\OperatorTok}[1]{\textcolor[rgb]{0.40,0.40,0.40}{{#1}}}
    \newcommand{\BuiltInTok}[1]{{#1}}
    \newcommand{\ExtensionTok}[1]{{#1}}
    \newcommand{\PreprocessorTok}[1]{\textcolor[rgb]{0.74,0.48,0.00}{{#1}}}
    \newcommand{\AttributeTok}[1]{\textcolor[rgb]{0.49,0.56,0.16}{{#1}}}
    \newcommand{\InformationTok}[1]{\textcolor[rgb]{0.38,0.63,0.69}{\textbf{\textit{{#1}}}}}
    \newcommand{\WarningTok}[1]{\textcolor[rgb]{0.38,0.63,0.69}{\textbf{\textit{{#1}}}}}
    
    
    % Define a nice break command that doesn't care if a line doesn't already
    % exist.
    \def\br{\hspace*{\fill} \\* }
    % Math Jax compatability definitions
    \def\gt{>}
    \def\lt{<}
    % Document parameters
    \title{Relat?rio-Copy2}
    
    
    

    % Pygments definitions
    
\makeatletter
\def\PY@reset{\let\PY@it=\relax \let\PY@bf=\relax%
    \let\PY@ul=\relax \let\PY@tc=\relax%
    \let\PY@bc=\relax \let\PY@ff=\relax}
\def\PY@tok#1{\csname PY@tok@#1\endcsname}
\def\PY@toks#1+{\ifx\relax#1\empty\else%
    \PY@tok{#1}\expandafter\PY@toks\fi}
\def\PY@do#1{\PY@bc{\PY@tc{\PY@ul{%
    \PY@it{\PY@bf{\PY@ff{#1}}}}}}}
\def\PY#1#2{\PY@reset\PY@toks#1+\relax+\PY@do{#2}}

\expandafter\def\csname PY@tok@w\endcsname{\def\PY@tc##1{\textcolor[rgb]{0.73,0.73,0.73}{##1}}}
\expandafter\def\csname PY@tok@c\endcsname{\let\PY@it=\textit\def\PY@tc##1{\textcolor[rgb]{0.25,0.50,0.50}{##1}}}
\expandafter\def\csname PY@tok@cp\endcsname{\def\PY@tc##1{\textcolor[rgb]{0.74,0.48,0.00}{##1}}}
\expandafter\def\csname PY@tok@k\endcsname{\let\PY@bf=\textbf\def\PY@tc##1{\textcolor[rgb]{0.00,0.50,0.00}{##1}}}
\expandafter\def\csname PY@tok@kp\endcsname{\def\PY@tc##1{\textcolor[rgb]{0.00,0.50,0.00}{##1}}}
\expandafter\def\csname PY@tok@kt\endcsname{\def\PY@tc##1{\textcolor[rgb]{0.69,0.00,0.25}{##1}}}
\expandafter\def\csname PY@tok@o\endcsname{\def\PY@tc##1{\textcolor[rgb]{0.40,0.40,0.40}{##1}}}
\expandafter\def\csname PY@tok@ow\endcsname{\let\PY@bf=\textbf\def\PY@tc##1{\textcolor[rgb]{0.67,0.13,1.00}{##1}}}
\expandafter\def\csname PY@tok@nb\endcsname{\def\PY@tc##1{\textcolor[rgb]{0.00,0.50,0.00}{##1}}}
\expandafter\def\csname PY@tok@nf\endcsname{\def\PY@tc##1{\textcolor[rgb]{0.00,0.00,1.00}{##1}}}
\expandafter\def\csname PY@tok@nc\endcsname{\let\PY@bf=\textbf\def\PY@tc##1{\textcolor[rgb]{0.00,0.00,1.00}{##1}}}
\expandafter\def\csname PY@tok@nn\endcsname{\let\PY@bf=\textbf\def\PY@tc##1{\textcolor[rgb]{0.00,0.00,1.00}{##1}}}
\expandafter\def\csname PY@tok@ne\endcsname{\let\PY@bf=\textbf\def\PY@tc##1{\textcolor[rgb]{0.82,0.25,0.23}{##1}}}
\expandafter\def\csname PY@tok@nv\endcsname{\def\PY@tc##1{\textcolor[rgb]{0.10,0.09,0.49}{##1}}}
\expandafter\def\csname PY@tok@no\endcsname{\def\PY@tc##1{\textcolor[rgb]{0.53,0.00,0.00}{##1}}}
\expandafter\def\csname PY@tok@nl\endcsname{\def\PY@tc##1{\textcolor[rgb]{0.63,0.63,0.00}{##1}}}
\expandafter\def\csname PY@tok@ni\endcsname{\let\PY@bf=\textbf\def\PY@tc##1{\textcolor[rgb]{0.60,0.60,0.60}{##1}}}
\expandafter\def\csname PY@tok@na\endcsname{\def\PY@tc##1{\textcolor[rgb]{0.49,0.56,0.16}{##1}}}
\expandafter\def\csname PY@tok@nt\endcsname{\let\PY@bf=\textbf\def\PY@tc##1{\textcolor[rgb]{0.00,0.50,0.00}{##1}}}
\expandafter\def\csname PY@tok@nd\endcsname{\def\PY@tc##1{\textcolor[rgb]{0.67,0.13,1.00}{##1}}}
\expandafter\def\csname PY@tok@s\endcsname{\def\PY@tc##1{\textcolor[rgb]{0.73,0.13,0.13}{##1}}}
\expandafter\def\csname PY@tok@sd\endcsname{\let\PY@it=\textit\def\PY@tc##1{\textcolor[rgb]{0.73,0.13,0.13}{##1}}}
\expandafter\def\csname PY@tok@si\endcsname{\let\PY@bf=\textbf\def\PY@tc##1{\textcolor[rgb]{0.73,0.40,0.53}{##1}}}
\expandafter\def\csname PY@tok@se\endcsname{\let\PY@bf=\textbf\def\PY@tc##1{\textcolor[rgb]{0.73,0.40,0.13}{##1}}}
\expandafter\def\csname PY@tok@sr\endcsname{\def\PY@tc##1{\textcolor[rgb]{0.73,0.40,0.53}{##1}}}
\expandafter\def\csname PY@tok@ss\endcsname{\def\PY@tc##1{\textcolor[rgb]{0.10,0.09,0.49}{##1}}}
\expandafter\def\csname PY@tok@sx\endcsname{\def\PY@tc##1{\textcolor[rgb]{0.00,0.50,0.00}{##1}}}
\expandafter\def\csname PY@tok@m\endcsname{\def\PY@tc##1{\textcolor[rgb]{0.40,0.40,0.40}{##1}}}
\expandafter\def\csname PY@tok@gh\endcsname{\let\PY@bf=\textbf\def\PY@tc##1{\textcolor[rgb]{0.00,0.00,0.50}{##1}}}
\expandafter\def\csname PY@tok@gu\endcsname{\let\PY@bf=\textbf\def\PY@tc##1{\textcolor[rgb]{0.50,0.00,0.50}{##1}}}
\expandafter\def\csname PY@tok@gd\endcsname{\def\PY@tc##1{\textcolor[rgb]{0.63,0.00,0.00}{##1}}}
\expandafter\def\csname PY@tok@gi\endcsname{\def\PY@tc##1{\textcolor[rgb]{0.00,0.63,0.00}{##1}}}
\expandafter\def\csname PY@tok@gr\endcsname{\def\PY@tc##1{\textcolor[rgb]{1.00,0.00,0.00}{##1}}}
\expandafter\def\csname PY@tok@ge\endcsname{\let\PY@it=\textit}
\expandafter\def\csname PY@tok@gs\endcsname{\let\PY@bf=\textbf}
\expandafter\def\csname PY@tok@gp\endcsname{\let\PY@bf=\textbf\def\PY@tc##1{\textcolor[rgb]{0.00,0.00,0.50}{##1}}}
\expandafter\def\csname PY@tok@go\endcsname{\def\PY@tc##1{\textcolor[rgb]{0.53,0.53,0.53}{##1}}}
\expandafter\def\csname PY@tok@gt\endcsname{\def\PY@tc##1{\textcolor[rgb]{0.00,0.27,0.87}{##1}}}
\expandafter\def\csname PY@tok@err\endcsname{\def\PY@bc##1{\setlength{\fboxsep}{0pt}\fcolorbox[rgb]{1.00,0.00,0.00}{1,1,1}{\strut ##1}}}
\expandafter\def\csname PY@tok@kc\endcsname{\let\PY@bf=\textbf\def\PY@tc##1{\textcolor[rgb]{0.00,0.50,0.00}{##1}}}
\expandafter\def\csname PY@tok@kd\endcsname{\let\PY@bf=\textbf\def\PY@tc##1{\textcolor[rgb]{0.00,0.50,0.00}{##1}}}
\expandafter\def\csname PY@tok@kn\endcsname{\let\PY@bf=\textbf\def\PY@tc##1{\textcolor[rgb]{0.00,0.50,0.00}{##1}}}
\expandafter\def\csname PY@tok@kr\endcsname{\let\PY@bf=\textbf\def\PY@tc##1{\textcolor[rgb]{0.00,0.50,0.00}{##1}}}
\expandafter\def\csname PY@tok@bp\endcsname{\def\PY@tc##1{\textcolor[rgb]{0.00,0.50,0.00}{##1}}}
\expandafter\def\csname PY@tok@fm\endcsname{\def\PY@tc##1{\textcolor[rgb]{0.00,0.00,1.00}{##1}}}
\expandafter\def\csname PY@tok@vc\endcsname{\def\PY@tc##1{\textcolor[rgb]{0.10,0.09,0.49}{##1}}}
\expandafter\def\csname PY@tok@vg\endcsname{\def\PY@tc##1{\textcolor[rgb]{0.10,0.09,0.49}{##1}}}
\expandafter\def\csname PY@tok@vi\endcsname{\def\PY@tc##1{\textcolor[rgb]{0.10,0.09,0.49}{##1}}}
\expandafter\def\csname PY@tok@vm\endcsname{\def\PY@tc##1{\textcolor[rgb]{0.10,0.09,0.49}{##1}}}
\expandafter\def\csname PY@tok@sa\endcsname{\def\PY@tc##1{\textcolor[rgb]{0.73,0.13,0.13}{##1}}}
\expandafter\def\csname PY@tok@sb\endcsname{\def\PY@tc##1{\textcolor[rgb]{0.73,0.13,0.13}{##1}}}
\expandafter\def\csname PY@tok@sc\endcsname{\def\PY@tc##1{\textcolor[rgb]{0.73,0.13,0.13}{##1}}}
\expandafter\def\csname PY@tok@dl\endcsname{\def\PY@tc##1{\textcolor[rgb]{0.73,0.13,0.13}{##1}}}
\expandafter\def\csname PY@tok@s2\endcsname{\def\PY@tc##1{\textcolor[rgb]{0.73,0.13,0.13}{##1}}}
\expandafter\def\csname PY@tok@sh\endcsname{\def\PY@tc##1{\textcolor[rgb]{0.73,0.13,0.13}{##1}}}
\expandafter\def\csname PY@tok@s1\endcsname{\def\PY@tc##1{\textcolor[rgb]{0.73,0.13,0.13}{##1}}}
\expandafter\def\csname PY@tok@mb\endcsname{\def\PY@tc##1{\textcolor[rgb]{0.40,0.40,0.40}{##1}}}
\expandafter\def\csname PY@tok@mf\endcsname{\def\PY@tc##1{\textcolor[rgb]{0.40,0.40,0.40}{##1}}}
\expandafter\def\csname PY@tok@mh\endcsname{\def\PY@tc##1{\textcolor[rgb]{0.40,0.40,0.40}{##1}}}
\expandafter\def\csname PY@tok@mi\endcsname{\def\PY@tc##1{\textcolor[rgb]{0.40,0.40,0.40}{##1}}}
\expandafter\def\csname PY@tok@il\endcsname{\def\PY@tc##1{\textcolor[rgb]{0.40,0.40,0.40}{##1}}}
\expandafter\def\csname PY@tok@mo\endcsname{\def\PY@tc##1{\textcolor[rgb]{0.40,0.40,0.40}{##1}}}
\expandafter\def\csname PY@tok@ch\endcsname{\let\PY@it=\textit\def\PY@tc##1{\textcolor[rgb]{0.25,0.50,0.50}{##1}}}
\expandafter\def\csname PY@tok@cm\endcsname{\let\PY@it=\textit\def\PY@tc##1{\textcolor[rgb]{0.25,0.50,0.50}{##1}}}
\expandafter\def\csname PY@tok@cpf\endcsname{\let\PY@it=\textit\def\PY@tc##1{\textcolor[rgb]{0.25,0.50,0.50}{##1}}}
\expandafter\def\csname PY@tok@c1\endcsname{\let\PY@it=\textit\def\PY@tc##1{\textcolor[rgb]{0.25,0.50,0.50}{##1}}}
\expandafter\def\csname PY@tok@cs\endcsname{\let\PY@it=\textit\def\PY@tc##1{\textcolor[rgb]{0.25,0.50,0.50}{##1}}}

\def\PYZbs{\char`\\}
\def\PYZus{\char`\_}
\def\PYZob{\char`\{}
\def\PYZcb{\char`\}}
\def\PYZca{\char`\^}
\def\PYZam{\char`\&}
\def\PYZlt{\char`\<}
\def\PYZgt{\char`\>}
\def\PYZsh{\char`\#}
\def\PYZpc{\char`\%}
\def\PYZdl{\char`\$}
\def\PYZhy{\char`\-}
\def\PYZsq{\char`\'}
\def\PYZdq{\char`\"}
\def\PYZti{\char`\~}
% for compatibility with earlier versions
\def\PYZat{@}
\def\PYZlb{[}
\def\PYZrb{]}
\makeatother


    % Exact colors from NB
    \definecolor{incolor}{rgb}{0.0, 0.0, 0.5}
    \definecolor{outcolor}{rgb}{0.545, 0.0, 0.0}



    
    % Prevent overflowing lines due to hard-to-break entities
    \sloppy 
    % Setup hyperref package
    \hypersetup{
      breaklinks=true,  % so long urls are correctly broken across lines
      colorlinks=true,
      urlcolor=urlcolor,
      linkcolor=linkcolor,
      citecolor=citecolor,
      }
    % Slightly bigger margins than the latex defaults
    
    \geometry{verbose,tmargin=1in,bmargin=1in,lmargin=1in,rmargin=1in}
    
    

    \begin{document}
    
    
    \maketitle
    
    

    
    \begin{Verbatim}[commandchars=\\\{\}]
{\color{incolor}In [{\color{incolor}1}]:} \PY{o}{\PYZpc{}}\PY{o}{\PYZpc{}}\PY{n+nx}{javascript}
        \PY{n+nx}{MathJax}\PY{p}{.}\PY{n+nx}{Hub}\PY{p}{.}\PY{n+nx}{Config}\PY{p}{(}\PY{p}{\PYZob{}}
            \PY{n+nx}{TeX}\PY{o}{:} \PY{p}{\PYZob{}} \PY{n+nx}{equationNumbers}\PY{o}{:} \PY{p}{\PYZob{}} \PY{n+nx}{autoNumber}\PY{o}{:} \PY{l+s+s2}{\PYZdq{}AMS\PYZdq{}} \PY{p}{\PYZcb{}} \PY{p}{\PYZcb{}}\PY{p}{,}
            \PY{n+nx}{tex2jax}\PY{o}{:} \PY{p}{\PYZob{}}
                \PY{n+nx}{inlineMath}\PY{o}{:} \PY{p}{[} \PY{p}{[}\PY{l+s+s1}{\PYZsq{}\PYZdl{}\PYZsq{}}\PY{p}{,}\PY{l+s+s1}{\PYZsq{}\PYZdl{}\PYZsq{}}\PY{p}{]}\PY{p}{,} \PY{p}{[}\PY{l+s+s2}{\PYZdq{}\PYZbs{}\PYZbs{}(\PYZdq{}}\PY{p}{,}\PY{l+s+s2}{\PYZdq{}\PYZbs{}\PYZbs{})\PYZdq{}}\PY{p}{]} \PY{p}{]}\PY{p}{,}
                \PY{n+nx}{displayMath}\PY{o}{:} \PY{p}{[} \PY{p}{[}\PY{l+s+s1}{\PYZsq{}\PYZdl{}\PYZdl{}\PYZsq{}}\PY{p}{,}\PY{l+s+s1}{\PYZsq{}\PYZdl{}\PYZdl{}\PYZsq{}}\PY{p}{]}\PY{p}{,} \PY{p}{[}\PY{l+s+s2}{\PYZdq{}\PYZbs{}\PYZbs{}[\PYZdq{}}\PY{p}{,}\PY{l+s+s2}{\PYZdq{}\PYZbs{}\PYZbs{}]\PYZdq{}}\PY{p}{]} \PY{p}{]}\PY{p}{,}
                \PY{n+nx}{processEscapes}\PY{o}{:} \PY{k+kc}{true}\PY{p}{,}
                \PY{n+nx}{processEnvironments}\PY{o}{:} \PY{k+kc}{true}
            \PY{p}{\PYZcb{}}\PY{p}{,}
            \PY{n+nx}{displayAlign}\PY{o}{:} \PY{l+s+s1}{\PYZsq{}center\PYZsq{}}\PY{p}{,} \PY{c+c1}{// Change this to \PYZsq{}center\PYZsq{} to center equations.     }
            \PY{l+s+s2}{\PYZdq{}HTML\PYZhy{}CSS\PYZdq{}}\PY{o}{:} \PY{p}{\PYZob{}}
                \PY{n+nx}{styles}\PY{o}{:} \PY{p}{\PYZob{}}\PY{l+s+s1}{\PYZsq{}.MathJax\PYZus{}Display\PYZsq{}}\PY{o}{:} \PY{p}{\PYZob{}}\PY{l+s+s2}{\PYZdq{}margin\PYZdq{}}\PY{o}{:} \PY{l+m+mi}{500}\PY{p}{\PYZcb{}}\PY{p}{\PYZcb{}}
            \PY{p}{\PYZcb{}}
        \PY{p}{\PYZcb{}}\PY{p}{)}\PY{p}{;}
\end{Verbatim}


    
    \begin{verbatim}
<IPython.core.display.Javascript object>
    \end{verbatim}

    
    \begin{Verbatim}[commandchars=\\\{\}]
{\color{incolor}In [{\color{incolor}2}]:} \PY{k+kn}{import} \PY{n+nn}{matplotlib}
        \PY{k+kn}{import} \PY{n+nn}{numpy} \PY{k}{as} \PY{n+nn}{np}
        
        \PY{o}{\PYZpc{}}\PY{k}{matplotlib}
\end{Verbatim}


    \begin{Verbatim}[commandchars=\\\{\}]
Using matplotlib backend: TkAgg

    \end{Verbatim}

    \hypertarget{primeira-questuxe3o}{%
\subsubsection{Primeira Questão}\label{primeira-questuxe3o}}

\hypertarget{a-modelo-matemuxe1tico}{%
\paragraph{a) Modelo matemático}\label{a-modelo-matemuxe1tico}}

Faremos então a dedução do primeiro problema tomando as seguintes
variáveis

\(Q\) - concentração\\
\(Z\) - vazão\\
\(V\) - volume\\
\(S\) - quantidade de sal

Para determinar a concentração \(Q(t)\) em um dado momento qualquer,
faremos \begin{equation}
    \label{eq_concentracao}
    Q(t) = \frac{S(t)}{V(t)}
\end{equation}

onde:

\begin{equation}
    \label{eq_volume}
    V(t) = V_{0} + (Z_{in} - Z_{out})t
\end{equation}

Sabemos que a variação da quantidade de sal \(dS(t)\) é dada pela
diferença entre a entrada e a saída de sal em um dado instante t, isto é

\begin{equation*}
    \label{eq_var_qtd_sal}
    dS(t) = S_{in} - S_{out}
\end{equation*}

\begin{equation*}
    \label{eq_var_qtd_sal_2}
    \frac{dS(t)}{dt} = \frac{dS_{in}}{dt} - \frac{dS_{out}}{dt}
\end{equation*}

\begin{equation*}
    \label{eq_var_qtd_sal_3}
    \frac{dS(t)}{dt} = \frac{d(Q_{in}(t) \cdot V_{in}(t))}{dt} - \frac{d(Q_{out}(t) \cdot V_{out}(t))}{dt}
\end{equation*}

\begin{equation*}
    \label{eq_var_qtd_sal_4}
    S'(t) = (Q'_{in}(t) \cdot V_{in}(t) + Q_{in}(t) \cdot V'_{in}(t)) - (Q'_{out}(t) \cdot V_{out}(t) + Q_{out}(t) \cdot V'_{out}(t))
\end{equation*}

Assumindo que a água que entra no tanque é pura, a concentração de
entrada \(Q_{in}\) é zero, então

\begin{equation*}
    \label{eq_var_qtd_sal_5}
    S'(t) = - (Q'_{out}(t) \cdot V_{out}(t) + Q_{out}(t) \cdot V'_{out}(t))
\end{equation*}

Uma vez que a massa de água que sai em um instante estacionário é nula,
podemos assumir que o termo \(Q'_{out}(t) \cdot V_{out}(t) = 0\), que
nos deixa com o seguinte resultado \begin{equation*}
    S'(t) = - Q_{out}(t) \cdot V'_{out}(t)
\end{equation*}

Contudo, sabemos que a vazão é dada por meio de: \begin{equation}
    \label{eq_vazao}
    Z(t) = V'(t)
\end{equation}

Assim, assumimos por meio de \eqref{eq_vazao} que \(S'(t)\) é dada pela
concentração de sal no tanque \(S(t)\) multiplicado pela vazão de saída
do tanque \(V_{out}\)

\begin{equation}
    \label{eq_var_qtd_sal_6}
    S'(t) = - Q(t) \cdot Z_{out}
\end{equation}

Substituindo \eqref{eq_concentracao} e \eqref{eq_volume} em
\eqref{eq_var_qtd_sal_6}, teremos:

\begin{equation}
    \label{eq_var_qtd_sal_7}
    S'(t) = - \frac{S(t)}{V(t)} \cdot Z_{out}
\end{equation}

Assim, nossa EDO será:

\begin{equation}
    \label{edo1}
    S'(t) + \frac{Z_{out}}{V_{0} + (Z_{in} - Z_{out})t} \cdot S(t) = 0
\end{equation}

Com a substituição dos valores dados na questão em \eqref{edo1},
teremos:

\begin{equation}
    \label{edo_com_valores}
    S'(t) + \frac{4}{100+2t} \cdot S(t) = 0
\end{equation}

\begin{center}\rule{0.5\linewidth}{\linethickness}\end{center}

Em posse dessa equação diferencial podemos realizar os procedimentos
para determinar qual sua solução exata para a quantidade de sal no
tanque \(S(t)\) em um dado instante \(t\). Para isto podemos resolver
como uma equação diferencial trivial: \begin{equation*}
    \frac{dS(t)}{dt} = - \frac{4}{100+2t} \cdot S(t) \therefore \frac{dS(t)}{S(t)} = - \frac{4dt}{100+2t}
\end{equation*} Segue que:

\begin{equation*}
    \int{\frac{dS(t)}{S(t)}} = - \int{\frac{4dt}{100+2t}}
\end{equation*}

\begin{equation}
    \label{edo_com_valores_sol}
    \ln(S(t)) = - 4\int{\frac{dt}{100+2t}}
\end{equation} A integral \(\int{\frac{4dt}{100+2t}}\) pode ser
resolvida por substituição fazendo: \begin{equation*}
    u = 100 + 2t \therefore du = 2dt \therefore dt = \frac{du}{2}
\end{equation*} Realizando a substituição em \eqref{edo_com_valores_sol}
\begin{align*}
    \ln(S(t)) &= - 4\int{\frac{du}{2u}} & \\
    \ln(S(t)) &= - 2\ln{u} &\\
    \ln(S(t)) &= - 2 \cdot \ln(100 + 2t) + C 
\end{align*}\\
Removendo o logarítmo, concluimos então que
\(S(t) = e^{C}\cdot(100 + 2t)^{-2}\)\\
Que pode ser reescrito como: \begin{equation}
    \label{edo_com_valores_sol_1}
    S(t) = \frac{C}{(100 + 2t)^{2}} \text{, onde C é uma constante arbitrária}
\end{equation} Como a quantidade de sal inicial no tanque de água é dada
por \(S(0) = 30\), podemos concluir que para esse PVI(Problema de valor
Inicial) a constante C será descrita por: \begin{align*}
    30 &= \frac{C}{(100 + 2 \cdot 0)^{2}}  & \\
    C &= 30 \cdot 100^{2} & \\ 
    C &= 3 \cdot 10^{5}
\end{align*}\\
O que nos deixa com a equação \begin{equation}
    \label{edo_com_valores_sol_2}
    S(t) = \frac{3 \cdot 10^{5}}{(100 + 2t)^{2}} \text{, onde C é uma constante arbitrária}
\end{equation}

\hypertarget{concentrauxe7uxe3o-de-sal-em-50-minutos}{%
\paragraph{Concentração de sal em 50
minutos}\label{concentrauxe7uxe3o-de-sal-em-50-minutos}}

E podemos aplicar essa equação para determinar o valor \(S(50)\):
\begin{equation*}
    \label{edo_com_valores_sol_3}
    S(t) = \frac{3 \cdot 10^{5}}{(100 + 2 \cdot 50)^{2}}
\end{equation*}

\begin{equation*}
    \label{edo_com_valores_sol_4}
    S(t) = \frac{3 \cdot 10^{5}}{4\cdot 10^{4}}
\end{equation*}

\begin{equation*}
    \label{edo_com_valores_sol_5}
    S(t) = 7.5 g
\end{equation*}

    \begin{center}\rule{0.5\linewidth}{\linethickness}\end{center}

\hypertarget{c-limite-no-infinito-para-esse-problema}{%
\paragraph{c) Limite no infinito para esse
problema}\label{c-limite-no-infinito-para-esse-problema}}

Aplicamos o limite no infinito sobre \eqref{edo_com_valores_sol_2} para
estudar o comportamento assintótico da função sobre o infinito, que
resulta em: \begin{equation*}
    \displaystyle{\lim_{t \to +\infty}} \frac{3 \cdot 10^{5}}{(100 + 2t)^{2}}
\end{equation*} Que é um limite polinomial clássico e que nos dá como
resultado que \begin{equation*}
    \displaystyle{\lim_{t \to +\infty}} \frac{3 \cdot 10^{5}}{(100 + 2t)^{2}} = 0
\end{equation*}\\
Essa solução nos permite interpretar que em um longo período de tempo o
sistema tende a alcançar o equilíbrio com a concentração de sal que
estava presente na água que entra, isto é, o sistema tende
assintóticamente a se tornar um sistema com água pura.

    \hypertarget{segunda-questuxe3o}{%
\subsubsection{Segunda Questão}\label{segunda-questuxe3o}}

\hypertarget{resolvendo-a-edo}{%
\paragraph{Resolvendo a EDO}\label{resolvendo-a-edo}}

Segundo a Lei de Resfriamento de Newton,

\begin{equation}
    \label{eq_newton}
    T' = -k \cdot (T - T_{a})
\end{equation}

Onde,\\
\(T\) - temperatura do corpo no instante t\\
\(T_{a}\) - temperatura constante do ambiente\\
\(t\) - tempo\\
\(k\) - constante

Resolvendo a EDO, teremos: \begin{equation*}
    \label{eq_newton_1}
    \frac{dT}{dt} = -k \cdot (T - T_{a})
\end{equation*}

\begin{equation*}
    \label{eq_newton_2}
    \frac{1}{T - T_{a}} \cdot dT = -k \cdot dt
\end{equation*}

Integrando os dois lados

\begin{equation*}
    \label{eq_newton_3}
    \int\frac{1}{T - T_{a}} \cdot dT = -\int k \cdot dt
\end{equation*}

\begin{equation*}
    \label{eq_newton_4}
    \ln(T - T_{a}) = - kt + C_{2} - C_{1}
\end{equation*}

Para tirar o \(\ln(T - T_{a})\), faremos

\begin{equation*}
    \label{eq_newton_5}
    e^{ln(T - T_{a})} = e^{-kt + c}
\end{equation*}

Logo,

\begin{equation*}
    T - T_{a} = e^{-kt} \cdot e^{c}
\end{equation*}

\begin{equation}
    \label{eq_newton_6}
    T - T_{a} = e^{-kt} \cdot C
\end{equation}

    \begin{center}\rule{0.5\linewidth}{\linethickness}\end{center}

\hypertarget{a-determinando-o-tempo-da-morte}{%
\paragraph{a) Determinando o tempo da
morte}\label{a-determinando-o-tempo-da-morte}}

Para encontrar o valor da constante \(C\), usaremos a informação dada de
que \(T(0) = 30^{\circ}C\) e \(T_{a} = 20^{\circ}C\), ou seja

\begin{equation*}
    \label{eq_achar_k_1}
    30 - 20 = e^{-k \cdot 0} \cdot C
\end{equation*}

\begin{equation}
    \label{eq_achar_k_2}
    C = 10
\end{equation}

Substituindo \eqref{eq_achar_k_2} em \eqref{eq_newton_6},
\begin{equation}
    \label{eq_newton_7}
    T - T_{a} = e^{-kt} \cdot 10
\end{equation}

A outra informação dada no problema nos permitirá encontrar a constante
\(k\):

Sabendo que em \(T(2) = 23^{\circ}C\),

\begin{equation*}
    \label{eq_newton_8}
    23 - 20 = e^{-2k} \cdot 10
\end{equation*}

\begin{equation*}
    \label{eq_newton_9}
    \frac{3}{10} = e^{-2k}
\end{equation*}

Aplicando \(ln\) nos dois lados,

\begin{equation*}
    \label{eq_newton_10}
    \ln(\frac{3}{10}) = \ln(e^{-2k})
\end{equation*}

\begin{equation*}
    \label{eq_newton_11}
    \ln(\frac{3}{10}) = -2k
\end{equation*}

Achamos \(k\):

\begin{equation*}
    k = -\frac{1}{2}\ln(\frac{3}{10})
\end{equation*}

\begin{equation}
    \label{eq_achou_k}
    k = \frac{1}{2}\ln(\frac{10}{3})
\end{equation}

Agora em posse dessas informações poderemos substituir a constante \(k\)
encontrada em \eqref{eq_newton_7}, deduzindo assim a equação final
\begin{equation}
    \label{eq_newton_final}
    T - T_{a} = 10e^{\frac{1}{2}\ln(\frac{10}{3})t}
\end{equation}

Agora que possuimos as constantes, poderemos determinar a hora
aproximada do crime substituindo os valores dados em
\eqref{eq_newton_final}:

\begin{equation*}
    \label{eq_newton_subst_1}
    37 - 20 = 10e^{-\frac{1}{2}\ln(\frac{10}{3})t}
\end{equation*}

\begin{equation*}
    \label{eq_newton_subst_2}
    17 = 10e^{\frac{1}{2}\ln(\frac{10}{3})t}
\end{equation*}

\begin{equation*}
    \label{eq_newton_subst_3}
    \frac{17}{10} = e^{\frac{1}{2}\ln(\frac{10}{3})t}
\end{equation*}

Aplicando \(ln\): \begin{equation*}
    \label{eq_newton_subst_4}
    \ln(\frac{17}{10}) = \frac{1}{2}\ln(\frac{10}{3})t
\end{equation*}

\begin{equation*}
    \label{eq_newton_subst_5}
    2\ln(\frac{17}{10}) = \ln(\frac{10}{3})t
\end{equation*}

\begin{equation*}
    \label{eq_newton_subst_6}
    2\ln(1.7) = \ln(10) - \ln(3)t
\end{equation*}

Encontramos t: \begin{equation*}
    \label{eq_newton_subst_7}
    t = 2\frac{\ln(1.7)}{\ln(10) - \ln(3)}
\end{equation*}

Sabemos que \(ln(1.7) \simeq 0.53\), \(ln(10) \simeq 2.3\) e
\(ln(3) = 1.1\), \begin{equation*}
    \label{eq_newton_subst_8}
    t = 2 \cdot\frac{0.53}{1.2}
\end{equation*}

\begin{equation*}
    \label{eq_newton_subst_9}
    t \simeq 0.88 \text{ horas}
\end{equation*}

Em minutos, \(t_{min} = t \cdot 60\), ou seja,

\begin{equation*}
    \label{eq_newton_subst_resp}
    t_{min} \simeq 53 \text{ minutos}
\end{equation*}

Sabemos que o corpo levou cerca de 53 minutos para sair de
\(37^{\circ}C\) para a temperatura de \(30^{\circ}C\), ou seja, a morte
ocorreu por volta de 53 minutos antes dele ter sido encontrado.

    \begin{center}\rule{0.5\linewidth}{\linethickness}\end{center}

\hypertarget{b-de-100circ-a-30circ}{%
\paragraph{\texorpdfstring{b) De \(100^{\circ}\) a
\(30^{\circ}\)}{b) De 100\^{}\{\textbackslash{}circ\} a 30\^{}\{\textbackslash{}circ\}}}\label{b-de-100circ-a-30circ}}

Consideramos esses valores para deduzir as questões conforme a solução
anterior:\\
\(T_{a} = 20^{\circ}C\)\\
\(T(0) = 100^{\circ}C\)\\
\(T(20) = 60^{\circ}C\)

Conforme o procedimento realizado anteriormente encontraremos o \(C\)
substituindo os valores de \(T(0)\) e \(T_{a}\) acima em
\eqref{eq_newton_6}

\begin{equation*}
    \label{eq_newton_subst_ec*}
    100 - 20 = e^{-k \cdot 0} \cdot C
\end{equation*}

\begin{equation}
    \label{eq_newton_achou_ec_b}
    C = 80
\end{equation}

Agora encontraremos \(k\) substituindo os valores de \(e^{c}\),
\(T(20)\) e \(T_{a}\) acima em \eqref{eq_newton_6}

\begin{equation*}
    \label{eq_newton_subst_k}
    60 - 20 = 80e^{-20k}
\end{equation*}

\begin{equation*}
    \label{eq_newton_subst_k1}
    1/2 = e^{-20k}
\end{equation*}

Aplicando \(ln\) nos dois lados da equação: \begin{equation*}
    \label{eq_newton_subst_k2}
    \ln(1/2) = -20k
\end{equation*}

Achamos o valor da constante \(k\):

\begin{equation*}
    k = -\frac{1}{20}\ln(1/2)
\end{equation*}

\begin{equation}
    \label{eq_newton_achou_k_b}
    k = \frac{1}{20}\ln(2)
\end{equation}

Substituindo os valores encontrados de \(T_{a}\), \(k\) e \(C\) em
\eqref{eq_newton_6}, obtemos a equação do corpo abaixo \begin{equation}
    \label{eq_newton_final_b}
    T - 20 = 80e^{-\frac{1}{20}\ln(2)t}
\end{equation}

O tempo necessário para a temperatura chegar 30˚C será dado por
\begin{equation*}
    \label{eq_newton_final_b_1}
    30 - 20 = 80e^{-\frac{1}{20}\ln(2)t}
\end{equation*}

\begin{equation*}
    \label{eq_newton_final_b_2}
    1/8 = e^{-\frac{1}{20}\ln(2)t}
\end{equation*}

Aplicando o \(ln\)

\begin{equation*}
    \label{eq_newton_final_b_3}
    \ln(1/8) = -\frac{1}{20}\ln(2)t
\end{equation*}

\begin{equation*}
    \label{eq_newton_final_b_4}
    t = -20 \cdot \frac{\ln(1/8)}{\ln(2)}
\end{equation*}

\begin{equation*}
    t = 20 \cdot \frac{\ln(8)}{\ln(2)}
\end{equation*}

\begin{equation*}
    t = 20 \cdot \frac{3 \cdot \ln(2)}{\ln(2)}
\end{equation*}

\begin{equation}
    \label{eq_newton_achou_t}
    t = 60 \text{ minutos}
\end{equation}

O que nos permite saber que o corpo leva cerca de 60 minutos para ir de
\(100^{\circ}C\) até \(30^{\circ}C\)

    \hypertarget{terceira-questuxe3o}{%
\subsubsection{Terceira Questão}\label{terceira-questuxe3o}}

\hypertarget{b-resolvendo-a-edo}{%
\paragraph{b) Resolvendo a EDO}\label{b-resolvendo-a-edo}}

Analisando a EDO dada na questão:

\begin{equation}
    \label{edo_3}
    V'(t) = \frac{2000 - 2V(t)}{200 - t}
\end{equation}

Teremos \begin{equation*}
    \label{edo_3_1}
    \frac{dV}{dt} = \frac{2000 - 2V(t)}{200 - t}
\end{equation*}

\begin{equation*}
    \label{edo_3_2}
    \frac{dV}{2000 - 2V} = \frac{dt}{200 - t}
\end{equation*}

Integrando os dois lados, teremos

\begin{equation*}
    \label{edo_3_3}
    \int\frac{1}{2000 - 2V}dV = \int\frac{1}{200 - t}dt
\end{equation*}

\begin{equation*}
    \label{edo_3_4}
    -\frac{1}{2}\ln(2000 - 2V) + C_{1} = -\ln(200 - t) + C_{2}
\end{equation*}

\begin{equation*}
    \label{edo_3_5}
    \frac{1}{2}\ln(2000 - 2V) = \ln(200 - t) + c
\end{equation*}

\begin{equation*}
    \label{edo_3_6}
    e^{\frac{1}{2}\ln(2000 - 2V)} = e^{\ln(200 - t) + c}
\end{equation*}

\begin{equation*}
    \label{edo_3_7}
    e^{\ln((2000 - 2V)^{\frac{1}{2}})} = e^{\ln(200 - t)}e^{c}
\end{equation*}

\begin{equation*}
    \label{edo_3_8}
    (2000 - 2V)^{\frac{1}{2}} = (200 - t) \cdot e^{c}
\end{equation*}

Elevando os dois lados a \(2\) para retirar o \(\frac{1}{2}\)

\begin{equation*}
    \label{edo_3_9}
    ((2000 - 2V)^{\frac{1}{2}})^{2} = ((200 - t) \cdot e^{c})^{2}
\end{equation*}

\begin{equation*}
    \label{edo_3_10}
    2000 - 2V = e^{2c}(200 - t)^{2}
\end{equation*}

\begin{equation*}
    V = \frac{2000 - e^{2c}(200 - t)^{2}}{2}
\end{equation*}

Assumindo que \(e^{ec} = C\)

\begin{equation}
    \label{edo_3_11}
    V = \frac{2000 - C(200 - t)^{2}}{2}
\end{equation}

Sabendo que \(V(0) = 0\), teremos

\begin{equation*}
    \label{edo_3_12}
    C(200)^{2} = 2000
\end{equation*}

\begin{equation*}
    C = \frac{2000}{40000}
\end{equation*}

\begin{equation}
    \label{edo_3_13}
    C = \frac{1}{20}
\end{equation}

Substituindo \eqref{edo_3_13} em \eqref{edo_3_11}

\begin{equation}
    \label{edo_final}
    V = \frac{2000 - \frac{1}{20} \cdot (200 - t)^{2}}{2}
\end{equation}

Usaremos a \eqref{edo_final} para calcular a velocidade exata para
quando \(t = 5\)

\begin{equation*}
    \label{edo_solucao_1}
    V = \frac{2000 - \frac{1}{20} \cdot(200 - 5)^{2}}{2}
\end{equation*}

\begin{equation*}
    \label{edo_solucao_2}
    V = \frac{2000 - 1901.25}{2}
\end{equation*}

\begin{equation}
    \label{edo_solucao_final}
    V = 49.375
\end{equation}


    % Add a bibliography block to the postdoc
    
    
    
    \end{document}
